This report presented the results of my internship project, which was focused on validating Oracle Linux virtualization modules. The main objective was to ensure the compatibility, stability, and reliability of key virtualization components, such as QEMU and Libvirt, across various configurations and software releases. Throughout the project, I undertook a comprehensive series of tests, including manual sanity testing and regression analysis, to achieve these objectives. \\

This project provided me with a valuable opportunity to apply the theoretical knowledge I gained during my studies at National Institute of Posts and Telecommunications, while also allowing me to explore and work with advanced virtualization technologies in a real-world environment. My time at Oracle has been an enriching experience, contributing to my technical expertise, problem-solving abilities, and teamwork skills. The hands-on experience I gained in documenting and analyzing test results, coupled with the challenges of adapting to new technologies, has greatly enhanced my understanding of virtualization and cloud computing. \\

One of the most significant aspects of this project was navigating the complex interactions between different software components and virtualization layers. Addressing the challenges associated with system updates and ensuring that new releases do not introduce regressions was particularly demanding. These experiences have further developed my ability to approach technical problems systematically and with a focus on continuous improvement. \\

Looking ahead, several perspectives could guide the future development and enhancement of Oracle Linux virtualization. Continued efforts could be directed towards further refining testing methodologies, particularly in the area of automation, to increase the efficiency and coverage of regression tests. Additionally, exploring the integration of more advanced security measures, such as enhanced encryption protocols and better isolation techniques, could further improve the robustness of the virtualization environment. By pursuing these avenues, Oracle Linux virtualization can continue to evolve into a more reliable and secure platform, meeting the growing demands of enterprise customers and ensuring seamless operation within the Oracle Cloud ecosystem.