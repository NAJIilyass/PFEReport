%%%%%%%%%%%%
% Resume %
%%%%%%%%%%%%

\begin{RLtext}

    يقدم هذا التقرير ملخصًا شاملاً لمشروع التدريب الذي تم إجراؤه في مركز أبحاث وتطوير \LR{Oracle} المغرب. كان الهدف من التدريب هو دراسة وتقييم إجراءات اختبار النظام والافتراضية في \LR{Oracle Linux} لضمان استقرار وأداء البيئات الافتراضية. كجزء من فريق اختبار الأجهزة والنظام والافتراضية في \LR{Oracle Linux}، شاركت في مهام مختلفة تتعلق باختبار \LR{QEMU} و \LR{Libvirt} عبر بيئات \LR{Oracle Linux} متعددة.

    ركز المشروع على تقييم التوافق والوظائف لوحدات الافتراضية، بما في ذلك \LR{QEMU} و \LR{Libvirt}، من خلال اختبارات دقيقة. شمل ذلك التحقق من أدائها تحت تكوينات وإصدارات برمجية مختلفة، بالإضافة إلى تنفيذ اختبارات التراجع لتحديد المشكلات المحتملة وحلها. تضمنت الأنشطة الرئيسية إجراء اختبارات يدوية على مضيفات \LR{Intel}و \LR{AMD}و \LR{ARM}، وتقييم سيناريوهات \LR{hotplug/unplug} لواجهات الشبكة، وتقييم فعالية عمليات \LR{hotplug} للذاكرة والتخزين.
    
    كانت نتائج هذه الاختبارات حاسمة لضمان أن تقنيات الافتراضية في \LR{Oracle Linux} تلبي معايير الأداء والموثوقية المطلوبة. من خلال تقديم رؤى مفصلة حول كل إجراء اختبار ونتائجه، يساهم التقرير في التحسين المستمر واستقرار حلول الافتراضية الخاصة بـ \LR{Oracle}، مما يدعم تكاملاً سلسًا وأداءً قويًا ضمن بيئات الحوسبة المتنوعة.
    \bigskip
\end{RLtext}

\noindent\rule{\linewidth}{0.3mm} \\[0.6cm] 
\begin{RLtext}
    \textbf{الكلمات المفتاحية:} \LR{Oracle Linux}، الافتراضية، \LR{QEMU}، \LR{Libvirt}، \LR{Oracle Cloud Infrastructure}، إجراءات الاختبار، اختبارات التراجع، تقييم الأداء.
\end{RLtext}

\noindent\rule{\linewidth}{0.3mm} \\[0.6cm]