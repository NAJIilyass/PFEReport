\begin{RLtext}

يستعرض هذا التقرير مساهماتي خلال التدريب النهائي المنجز لدى مجموعة بوسطن الاستشارية - وحدة التسليم \LR{X (BCG X)}، حيث قمت بتطوير منصة مبتكرة لإدارة العقود الذكية اعتمادًا على تقنيات الذكاء الاصطناعي التوكيلية \LR{(Agentic AI)}. تم تنفيذ المشروع في إطار حصولي على دبلوم مهندس دولة في الاتصالات وتكنولوجيا المعلومات، تخصص هندسة البرمجيات المتقدمة للخدمات الرقمية، بهدف أتمتة وتحسين العمليات القانونية التقليدية التي تعتمد بشكل كبير على التدخل اليدوي.

تم تنظيم دورة حياة المشروع بطريقة منهجية تضمنت تحليل المتطلبات، والمقارنة المرجعية للحلول الحالية، والتصميم المعماري المفصل، والتنفيذ الدقيق، والتحقق الشامل. شملت التقنيات الأساسية المستخدمة \LR{Azure OpenAI}، و\LR{LangChain}، و\LR{LangGraph}، و\LR{Elasticsearch}، و\LR{React}، و\LR{FastAPI}.

خلال فترة التدريب، قمت بتطوير مهاراتي التقنية والإدارية بشكل ملحوظ، خصوصًا في التغلب على التحديات المعقدة المتعلقة بدمج الذكاء الاصطناعي، وإدارة التعاون في الوقت الفعلي، وضمان الأمن والامتثال. مستقبلًا، يمكن تعزيز المنصة عبر دمج التحليلات التنبؤية، وزيادة النمطية لضمان المرونة والتوسع، وتقوية الإجراءات الأمنية، وتحسين مستمر لتجربة المستخدم، مما يجعل هذه المنصة محركًا رئيسيًا للتحول الرقمي في العمليات القانونية.

\bigskip
\end{RLtext}

\noindent\rule{\linewidth}{0.3mm} \\[0.05cm] 
\begin{RLtext}
    \textbf{الكلمات المفتاحية:} التحول الرقمي، إدارة العقود الذكية،\LR{Agentic AI}،\LR{LangGraph}،\LR{Azure OpenAI}،\LR{.Elasticsearch}
\end{RLtext}

\noindent\rule{\linewidth}{0.3mm} \\[0.6cm]