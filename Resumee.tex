\begin{RLtext}

يستعرض هذا التقرير مساهماتي خلال التدريب النهائي المنجز لدى مجموعة بوسطن الاستشارية - وحدة التسليم \LR{(BCG X)}، حيث قمت بتطوير منصة مبتكرة لإدارة العقود الذكية اعتمادًا على تقنيات الذكاء الاصطناعي التوكيلية \LR{(Agentic AI)}. تم تنفيذ المشروع في إطار حصولي على دبلوم مهندس دولة في الاتصالات وتكنولوجيا المعلومات، تخصص هندسة البرمجيات المتقدمة للخدمات الرقمية، بهدف أتمتة وتحسين عمليات مثل صياغة العقود، والتحقق منها، وتتبع الامتثال—وهي مهام قانونية متكررة وتعتمد بشكل كبير على الجهد اليدوي.

يستهدف المشروع التحديات الأوسع المتعلقة بعدم الكفاءة ومخاطر الامتثال في العمليات القانونية، والتي غالبًا ما تنتج عن التواصل المجزأ وتعقيد الأنظمة التنظيمية. ونظرًا للحاجة إلى حل قانوني ذكي وقابل للتوسع، تقدم المنصة المقترحة نظامًا ذاتيًا وتكيفيًا يُقلل المخاطر، ويُحسِّن الدقة، ويُسرِّع من دورات معالجة العقود.

بالإضافة إلى تسليم منصة وظيفية، عزز هذا المشروع مهاراتي في تكامل تقنيات الذكاء الاصطناعي، والتعاون الآمن، وتصميم الأنظمة المعقدة. ويمكن أن تسهم تطوراتها المستقبلية في تعزيز التحول الرقمي للعمليات القانونية من خلال تحليلات تنبؤية، ونمطية أقوى، وتحسينات مستمرة متمحورة حول المستخدم.

\bigskip
\end{RLtext}

\noindent\rule{\linewidth}{0.3mm} \\[0.05cm] 
\begin{RLtext}
    \textbf{الكلمات المفتاحية:} التحول الرقمي، إدارة العقود الذكية،\LR{Agentic AI}،\LR{LangGraph}،\LR{Azure OpenAI}،\LR{.Elasticsearch}
\end{RLtext}

\noindent\rule{\linewidth}{0.3mm} \\[0.6cm]