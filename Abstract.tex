%%%%%%%%%%%%
% Abstract %
%%%%%%%%%%%%


This report offers a comprehensive summary of the internship project undertaken at Oracle Morocco Research \& Development Center. The internship aimed to investigate and validate Oracle Linux Virtualization and System Testing procedures to ensure the stability and performance of virtualized environments. As part of the Oracle Linux Virtualization, System, and Hardware Testing team, I engaged in various tasks involving QEMU and Libvirt testing across multiple Oracle Linux environments. \\

The project's focus was on assessing the compatibility and functionality of virtualization modules, including QEMU and Libvirt, through rigorous testing. This involved validating their performance under different configurations and software versions, as well as implementing regression tests to identify and resolve potential issues. Key activities included manual sanity tests on Intel, AMD and ARM hosts, evaluating network interface hotplug/unplug scenarios, and assessing the effectiveness of memory and disk hotplug operations. \\

The outcomes of these tests were instrumental in ensuring that Oracle Linux virtualization technologies met the required performance and reliability standards. By providing detailed insights into each testing procedure and its results, the report contributes to the ongoing improvement and stability of Oracle's virtualization solutions, supporting a seamless integration and robust performance within diverse computing environments.
\bigskip

\noindent\rule{\textwidth}{0.3mm} \\[0.6cm] 
\textbf{Key words:} 
Oracle Linux, Virtualization, QEMU, Libvirt, Oracle Cloud Infrastructure, Testing Procedures, Regression Testing, Performance Evaluation
\\
\noindent\rule{\textwidth}{0.3mm} \\[0.6cm]