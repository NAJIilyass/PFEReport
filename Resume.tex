Ce rapport présente mes contributions lors de mon stage de fin d’études chez Boston Consulting Group - X Delivery (BCG X), portant sur le développement d’une plateforme innovante de gestion intelligente des contrats utilisant des technologies d’intelligence artificielle agentique (Agentic AI). Réalisé dans le cadre du diplôme d’Ingénieur d’État en Télécommunications et Technologies de l’Information, avec une spécialisation en Ingénierie logicielle avancée pour les services numériques, le projet visait à automatiser et optimiser des processus tels que la rédaction, la validation et le suivi de conformité des contrats—des tâches juridiques souvent répétitives et manuelles.\mynewline

Le projet s’attaque aux problématiques plus larges d’inefficacité et de risques de non-conformité dans les processus juridiques, souvent causés par des communications fragmentées et une complexité réglementaire. Motivé par la nécessité d’une solution juridique intelligente et évolutive, la plateforme proposée exploite l’Agentic AI pour offrir un système autonome et adaptatif, capable de réduire les risques, d’améliorer la précision et d’accélérer les cycles contractuels.\mynewline

Au-delà du développement de la plateforme, ce projet m’a permis de renforcer mes compétences en intégration de l’IA, en collaboration sécurisée et en conception de systèmes complexes. Son évolution future pourrait amplifier la transformation numérique des opérations juridiques grâce à l’analytique prédictive, à une modularité accrue et à une amélioration continue centrée sur l’utilisateur.

\bigskip

\noindent\rule{\linewidth}{0.3mm} \\[0.4cm] 
\textbf{Mots clés :}
Gestion des Contrats, Agentic AI, LangGraph, Azure OpenAI, Elasticsearch, Transformation Digitale
\\[0.1cm]
\noindent\rule{\linewidth}{0.3mm} \\[0.6cm] 
