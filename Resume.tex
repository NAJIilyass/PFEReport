%%%%%%%%%%%%
% Resume %
%%%%%%%%%%%%

Ce rapport présente le résumé du projet de stage d'application entrepris au Centre de Recherche \& Développement d'Oracle Maroc. Le stage avait pour objectif d'étudier et de valider les procédures de virtualisation et de test des systèmes Oracle Linux afin d'assurer la stabilité et les performances des environnements virtualisés. Au sein de l'équipe de virtualisation, de systèmes et de tests matériels d'Oracle Linux, j'ai participé à diverses tâches impliquant des tests de QEMU et Libvirt sur plusieurs environnements Oracle Linux. \\

Le projet se concentrait sur l'évaluation de la compatibilité et de la fonctionnalité des modules de virtualisation, y compris QEMU et Libvirt, à travers des tests rigoureux. Cela impliquait de valider leurs performances sous différentes configurations et versions de logiciels, ainsi que de mettre en œuvre des tests de régression pour identifier et résoudre d'éventuels problèmes. Les activités clés comprenaient des tests manuels de validation sur des hôtes Intel, AMD et ARM, l'évaluation des scénarios de hotplug/unplug des interfaces réseau, ainsi que l'évaluation de l'efficacité des opérations de hotplug de mémoire et de disque. \\

Les résultats de ces tests ont été essentiels pour garantir que les technologies de virtualisation d'Oracle Linux répondaient aux normes de performance et de fiabilité requises. En fournissant des informations détaillées sur chaque procédure de test et ses résultats, le rapport contribue à l'amélioration continue et à la stabilité des solutions de virtualisation d'Oracle, soutenant une intégration transparente et des performances robustes dans divers environnements informatiques.
\bigskip

\noindent\rule{\linewidth}{0.3mm} \\[0.6cm] 
\textbf{Mots clés :}
Oracle Linux, Virtualisation, QEMU, Libvirt, Oracle Cloud Infrastructure, Procédures de test, Tests de régression, Évaluation des performances
\\
\noindent\rule{\linewidth}{0.3mm} \\[0.6cm] 
